\documentclass[a4paper,10pt]{article}
\usepackage[paper=a4paper, hmargin=1.5cm, bottom=1.5cm, top=3.5cm]{geometry}
\usepackage[latin1]{inputenc}
\usepackage[T1]{fontenc}
\usepackage[spanish]{babel}
\usepackage{xspace}
\usepackage{xargs}
\usepackage{ifthen}
\usepackage{aed2-tad,aed2-symb,aed2-itef}
\usepackage{textcomp}
\usepackage{framed}

\input{macrosAlgo1}

\newcommand{\moduloNombre}[1]{\textbf{#1}}

\let\NombreFuncion=\textsc
\let\TipoVariable=\texttt
\let\ModificadorArgumento=\textbf
\newcommand{\res}{$res$\xspace}
\newcommand{\tab}{\hspace*{7mm}}

\newcommandx{\TipoFuncion}[3]{%
  \NombreFuncion{#1}(#2) \ifx#3\empty\else $\to$ \res\,: \TipoVariable{#3}\fi%
}
\newcommand{\In}[2]{\ModificadorArgumento{in} \ensuremath{#1}\,: \TipoVariable{#2}\xspace}
\newcommand{\Out}[2]{\ModificadorArgumento{out} \ensuremath{#1}\,: \TipoVariable{#2}\xspace}
\newcommand{\Inout}[2]{\ModificadorArgumento{in/out} \ensuremath{#1}\,: \TipoVariable{#2}\xspace}
\newcommand{\Aplicar}[2]{\NombreFuncion{#1}(#2)}

\newlength{\IntFuncionLengthA}
\newlength{\IntFuncionLengthB}
\newlength{\IntFuncionLengthC}
%InterfazFuncion(nombre, argumentos, valor retorno, precondicion, postcondicion, complejidad, descripcion, aliasing)
\newcommandx{\InterfazFuncion}[9][4=true,6,7,8,9]{%
  \hangindent=\parindent
  \TipoFuncion{#1}{#2}{#3}\\%
  \textbf{Pre} $\equiv$ \{#4\}\\%
  \textbf{Post} $\equiv$ \{#5\}%
  \ifx#6\empty\else\\\textbf{Complejidad:} #6\fi%
  \ifx#7\empty\else\\\textbf{Descripci�n:} #7\fi%
  \ifx#8\empty\else\\\textbf{Aliasing:} #8\fi%
  \ifx#9\empty\else\\\textbf{Requiere:} #9\fi%
}

\newenvironment{Interfaz}{%
  \parskip=2ex%
  \noindent\textbf{\Large Interfaz}%
  \par%
}{}

\newenvironment{Representacion}{%
  \vspace*{2ex}%
  \noindent\textbf{\Large Representaci�n}%
  \vspace*{2ex}%
}{}

\newenvironment{Algoritmos}{%
  \vspace*{2ex}%
  \noindent\textbf{\Large Algoritmos}%
  \vspace*{2ex}%
}{}


\newcommand{\Titulo}[1]{
  \vspace*{1ex}\par\noindent\textbf{\large #1}\par
}

\newenvironmentx{Estructura}[2][2={estr}]{%
  \par\vspace*{2ex}%
  \TipoVariable{#1} \textbf{se representa con} \TipoVariable{#2}%
  \par\vspace*{1ex}%
}{%
  \par\vspace*{2ex}%
}%

\newboolean{EstructuraHayItems}
\newlength{\lenTupla}
\newenvironmentx{Tupla}[1][1={estr}]{%
    \settowidth{\lenTupla}{\hspace*{3mm}donde \TipoVariable{#1} es \TipoVariable{tupla}$($}%
    \addtolength{\lenTupla}{\parindent}%
    \hspace*{3mm}donde \TipoVariable{#1} es \TipoVariable{tupla}$($%
    \begin{minipage}[t]{\linewidth-\lenTupla}%
    \setboolean{EstructuraHayItems}{false}%
}{%
    $)$%
    \end{minipage}
}

\newcommandx{\tupItem}[3][1={\ }]{%
    %\hspace*{3mm}%
    \ifthenelse{\boolean{EstructuraHayItems}}{%
        ,#1%
    }{}%
    \emph{#2}: \TipoVariable{#3}%
    \setboolean{EstructuraHayItems}{true}%
}

\newcommandx{\RepFc}[3][1={estr},2={e}]{%
  \tadOperacion{Rep}{#1}{bool}{}%
  \tadAxioma{Rep($#2$)}{#3}%
}%

\newcommandx{\Rep}[3][1={estr},2={e}]{%
  \tadOperacion{Rep}{#1}{bool}{}%
  \tadAxioma{Rep($#2$)}{true \ssi #3}%
}%

\newcommandx{\Abs}[5][1={estr},3={e}]{%
  \tadOperacion{Abs}{#1/#3}{#2}{Rep($#3$)}%
  \settominwidth{\hangindent}{Abs($#3$) \igobs #4: #2 $\mid$ }%
  \addtolength{\hangindent}{\parindent}%
  Abs($#3$) \igobs #4: #2 $\mid$ #5%
}%

\newcommandx{\AbsFc}[4][1={estr},3={e}]{%
  \tadOperacion{Abs}{#1/#3}{#2}{Rep($#3$)}%
  \tadAxioma{Abs($#3$)}{#4}%
}%


\newcommand{\DRef}{\ensuremath{\rightarrow}}

\begin{document}

\begin{tabbing}
VectorDeLetras = \= $< letras(26 \ punteros \ a \ VectorDeLetras), $\\
\>$ \ resultado:  *Conj(\alpha) >$ \\

\end{tabbing}

\section*{M�dulo Trie($\alpha$)}


\begin{Interfaz}
  
  \textbf{par�metros formales}
  
  \textbf{g�neros}: $\alpha$.

  \textbf{se explica con}: Trie.

  \Titulo{Operaciones b�sicas de trie}

  \InterfazFuncion{Iniciar}{}{trie}%
  {$res \igobs Iniciar()$}%
  [$\Theta(1)$]
  [genera unn trie vac�o.]

  \InterfazFuncion{AgregarResultado}{\Inout{f}{trie}, \In{s}{string}, \In{a}{$\alpha$}}{}
  [$f \igobs f_0$]
  {$f \igobs AgregarResultado(f_0,s,t)$}
  [$\Theta(long(a))$]
  [agrega la palabra para despues poder buscarla.]
  [el elemento $s$ agrega por copia. $f$ se pasa por referencia.]

 
  \InterfazFuncion{Buscar}{\Inout{f}{trie}, \In{s}{string}}{conj($\alpha$)}
  [$(\forall i \leftarrow [ 0..\textbar a \textbar ) a_i \in \rango{a}{z} $ ]
  {$res \igobs Buscar(f,s)$}
  [$\Theta(long(a))$]
  [devuelve un puntero al conjunto que coincide con la b�squeda.]
  [el elemento $s$ agrega por copia. $f$ se pasa por referencia.]


\end{Interfaz}

\begin{Representacion}
  
  \Titulo{Representaci�n del trie}

  \begin{Estructura}{trie}[trie]
    \begin{Tupla}[trie]
      \tupItem{principal}{VectorDeLetras : < letras(26 punteros a VectorDeLetras), resultado: *Conj($\alpha$) >}%
	  ~\\      
      \tupItem{conjuntos}{Conj(Conj($\alpha$))}%
	  ~\\	      
      \tupItem{arreglos}{conj(VectorDeLetras : < letras(26 punteros a VectorDeLetras), resultado: *Conj($\alpha$) >)}
    \end{Tupla}

  \end{Estructura}


 

\end{Representacion}

\begin{Algoritmos}
	\begin{tabbing}
	
	Iniciar():\= \\
	\> $res.conjnutos \leftarrow *vacio$ \\
	\> $res.arreglos \leftarrow *vacio $\\	
	\> $res.principal \leftarrow <(NULL, ... , NULL),(NULL)>$
	\end{tabbing}
	~
	~\\
	\begin{tabbing}
	letraToNum \= $( \In{s}{string}): \rightarrow res: int$ \\	
	\>	$res \leftarrow toAscii(s) -65$
	\end{tabbing}
	~
	\begin{tabbing}
	agregar\= Resultado \=  $(\Inout{f}{trie},\In{s}{string}, \In{t}{$\alpha$}):$ \\
	\> $cant \leftarrow Longitud(s)$ \\
	\> $currentDic \leftarrow f.principal $ \\
	\> $for  \ i \leftarrow 1 \ to \ cant$\\
	\> \> $letraActual \leftarrow = s[i] $\\
	\> \> $letNum \leftarrow letraToNum(letraActual) $ \\
	\>\> $if currentDic.$ \= $letras[letNum] == NULL $\\
	\>\>\> $ NuevasLetras \leftarrow \ <(NULL, .., NULL), (NULL)> $ \\
	\>\>\> $f.arreglos  \leftarrow Agregar(f.arreglos, NuevasLetras) $ \\
	\>\>\> $currentDic[letNum] = *NuevasLetras $ \\
	\>\>\> $currentDic = *currentDic.letras[letNum] $ \\
	\>\> $else$ \\
	\>\>\> $currentDic = *currentDic.letras[letNum] $ \\
	\>\> $ fi$ \\
	\> $endfor$ \\
	~\\
	\>$ if currentDic.conjuntos.resultado == NULL  $\\
	\>\>$ nuevoConjTema \leftarrow \ *vacio $\\
	\>\> $ nuevoConjTema \leftarrow \ Agregar( nuevoConjTema, t) $ \\
	\>\> $ f.conjuntos \leftarrow \ Agregar(f.conjuntos, nuevoConjTema) $ \\
	\>\> $ currentDic.temas.resultado \leftarrow \& nuevoConjTema $ \\
	\>$ else	$ \\
	\>\> $ currentDic.temas.resultado \leftarrow \ Agregar (\& currentDic.conjuntos.resultado,t) $ \\
	\>$fi$ \\
	\end{tabbing}
	~\\
	\begin{tabbing}
	
		
	Buscar(\= \In{f}{trie}, \In{s}{string}) $\rightarrow$ res: *Conj($\alpha$) \\
	~\\
	\> $cant \leftarrow \ Longitud(s) $ \\
	\> $currentDic = f.principal $\\
    ~
    \> $ for \ i $\= $ \leftarrow \ 1 \ to \ cant $\\
    \>\> $letraActual \leftarrow \ s[i] $ \\
    \>\> $letNum \leftarrow \ letraToNum(letraActual) $\\
    ~
    \>\> $if $ \= $(currentDic.letras[letNum] == NULL) $\\
    \>\>\> $ res \leftarrow \ *vacio $ \\
    \>\> $else$ \\
    \>\>\> $currentDic \leftarrow *currentDic.letras[letNum] $ \\
    \>\> $fi$ \\
    \> $endfor$ \\
    \> $res \leftarrow  *currentDic.resultado$ \\
    
	\end{tabbing}	
	
\end{Algoritmos}
\end{document}
